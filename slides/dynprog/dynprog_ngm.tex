\documentclass[11pt,xcolor={dvipsnames},aspectratio=159,hyperref={pdftex,pdfpagemode=UseNone,hidelinks,pdfdisplaydoctitle=true},usepdftitle=false]{beamer}
\usepackage{presentation}[aspectratio=169]
\usepackage{math}
\usepackage{mathtools}
\usepackage{mleftright}
\usepackage{algorithm}% http://ctan.org/pkg/algorithms
\usepackage{algpseudocode}% http://ctan.org/pkg/algorithmicx
\hypersetup{
    colorlinks=magenta,
    linkcolor=magenta,
    filecolor=magenta,      
    urlcolor=magenta,
    }
\hypersetup{pdftitle={NGM}}
% Enter link to PDF file with figures:

\begin{document}
% Enter presentation title:
\title{Continuous state variables: Neoclassical Growth Model}
\subtitle{Quantitative Economics 2024}
% Enter presentation information:

% Enter presentation authors:
\author{Piotr Żoch}%
% Enter presentation location and date (optional; comment line if not needed):
\frame{\titlepage}

% Fill out content of presentation:
\begin{frame}{Motivation}   
   
\begin{itemize}
    \item We analyzed finite \al{Markov decision processes} (MDP).
    \begin{enumerate}
        \item \text{X} is a finite set: \al{state space}.
        \item \text{A} is a finite set: \al{action space}.
    \end{enumerate}
    \item In many interesting applications, \al{state space} and \al{action space} are \alb{not finite}. 
    \item Example: a household chooses how much to consume and how much to save: real numbers.
    \item Instead of studying general results I will focus on a specific example -- \alg{neoclassical growth model} -- and illustrate some computational issues.
    
\end{itemize}
\end{frame}

\begin{frame}{Neoclassical Growth Model}   
 \begin{itemize}
    \item The planner maximizes \begin{align*}
        \E_0 \sum_{t=0}^\infty \b^t u\of{c_t}, \quad \text{ subject to } \quad k_{t+1} = z_t f\of{k_t}  + \bp{1-\d} k_t - c_t \text{ for all } t \geq 0
    \end{align*}
    \item We restrict $k_t, c_t \geq 0$ for all $t \geq 0$. $k_0$ is given. 
    \item $\beta \in \bp{0,1}$ is the discount factor, $\delta \in (0,1]$ is the depreciation rate.
    \item $u$ is a \al{utility function} and $f$ is a \al{production function}.
    \item $z_t>0$ is a \alb{productivity shock} that follows a Markov chain with values in $\text{Z}$. For simplicity assume $\text{Z}$ is finite and $P$ is a transition matrix.
\end{itemize}
\end{frame}

\begin{frame}{Neoclassical Growth Model}   
    \begin{itemize}
       \item We assume \begin{itemize}
        \item $u: \R_+ \rightarrow \R$ is continuous, strictly increasing, strictly concave and twice continuously differentiable with \begin{align*} \lim_{c\rightarrow 0} u^\prime\of{c} = \infty, \; \lim_{c\rightarrow \infty} u^\prime\of{c} = 0. \end{align*}
        \item $f: \R_+ \rightarrow \R$ is continuous, strictly increasing, strictly concave and twice continuously differentiable with  \begin{align*} \lim_{k\rightarrow 0} f^\prime\of{k} = \infty, \; \lim_{k\rightarrow \infty} f^\prime\of{k} = 0, \; f\of{0} = 0. \end{align*}
       \end{itemize}
       \item These assumptions will allow us to ignore corner solutions.
   \end{itemize}
   \end{frame}
   

   \begin{frame}{Neoclassical Growth Model}   
    \begin{itemize}
       \item The state variables are \begin{align*}
           x_t \coloneq \bp{k_t, z_t} \in \text{X} \coloneq \R_+ \times \text{Z}.
       \end{align*}
    \item The state space is \alr{not} finite.
    \item Let action $a$ be the choice of capital next period: \begin{align*}
        a_t \coloneq k_{t+1} \in \text{A} \coloneq \R_+.
    \end{align*}
    \item The action space is \alr{not} finite.
    \item It is also possible to associate $a_t$ with consumption $c_t$.
   \end{itemize}
   \end{frame}

\begin{frame}{Neoclassical Growth Model}
    \begin{itemize}
        \item Bellman equation is \begin{align*}
            V\of{k,z} = \max_{k^\prime \in \G\of{k,z}} \bc{u\of{z f\of{k} - k^\prime + \bp{1-\delta}k} + \b \sum_{z^\prime \in \text{Z}} P\of{z,z^\prime} V\of{k^\prime,z^\prime} },
        \end{align*}
        where \begin{align*}\Gamma\of{k,z} =\bs{0,zf\of{k} + \bp{1-\d}k}.\end{align*}
    \end{itemize}
    \end{frame}

    \begin{frame}{Neoclassical Growth Model}
        \begin{itemize}
            \item \alr{Problem:} things we learned about finite state MDPs do not apply here!
            \item Easier part: theorems that do not require finiteness of state space and action space (see Stokey and Lucas, 1989).
            \item Harder part: how to solve it numerically in an efficient way?
        \end{itemize}
        \end{frame}

    \begin{frame}
        \heading{Discretization}
        \end{frame}


\begin{frame}{Discretization}   
    \begin{itemize}
        \item One possible idea is to discretize the state space and action space and then solve the problem numerically using the methods we learned for finite state MDPs.
        \item We know we will find an exact solution to the \al{discretized problem}, but will it be a good approximation to the original problem?
        \item We first need to choose a \al{grid} for $k$ and $z$.
        \item We focus on $k$ -- assume that $z$ is finite state Markov chain (if it is not, we can discretize it as well using Tauchen or Rouwenhurst method).
    \end{itemize}
    \end{frame}


\begin{frame}{Discretization}   
    \begin{itemize}
        \item First step: choose $k_{\min}$ and $k_{\max}$ and the number of points $N_k$. 
        \item This often requires some experimentation. \begin{itemize}
            \item Too many points -- it will make the problem costly to solve. 
            \item Too few points -- it will make the solution to the discretized problem far from the true solution of the original problem.
            \item The interval $\bs{k_{\min}, k_{\max}}$ too wide -- it will make the problem costly to solve.
            \item The interval $\bs{k_{\min}, k_{\max}}$ too narrow -- it will impose a constraint that was not present in the original problem.
        \end{itemize}
    \end{itemize}
    \end{frame}


\begin{frame}{Discretization}   
    \begin{itemize}
        \item Do not choose $k_{\min}=0$: recall that $f\of{0}=0$, so if $\lim_{c\rightarrow 0 } u\of{c} = -\infty$ then you will run into problems.
        \item To choose $k_{\max}$, note that because $\lim_{k\rightarrow \infty} f^\prime\of{k} = 0$ for each $z$ there exists a $k_{\max,z}$ such that $zf\of{k}+\bp{1-\d}k<k$ for all $k>k_{\max,z}$.
    \end{itemize}
    \end{frame}
    
    
\begin{frame}{Residuals}   
    \begin{itemize}
        \item How to verify that the solution to the discretized problem is a good approximation to the solution of the original problem?
        \item Recall the RHS of the Bellman equation: \begin{align*}
            \max_{k^\prime \in \G\of{k,z}} \bc{u\of{z f\of{k} + \bp{1-\d}k - k^\prime} + \b \sum_{z^\prime \in \text{Z}} P\of{z,z^\prime} V\of{k^\prime,z^\prime} }.
        \end{align*}
        \item The first order condition with respect to $k^\prime$ is \begin{align*}
            - u^\prime\of{z f\of{k} \bp{1-\d}k - k^\prime} + \b \sum_{z^\prime \in \text{Z}} P\of{z,z^\prime} V^\prime\of{k^\prime,z^\prime} = 0.
        \end{align*}
        where $V^\prime\of{k^\prime,z^\prime}$ is the derivative of $V$ with respect to $k^\prime$.
    \end{itemize}
    \end{frame}
            
\begin{frame}{Residuals}   
    \begin{itemize}
        \item Evaluate the LHS of the first order condition at the solution to the discretized problem at various $k$ (including these that did not belong to the grid) and $z$.
        \item If the solution is a good approximation, the LHS should be close to zero.
        \item What does "close" mean?
    \end{itemize}
    \end{frame}

\begin{frame}{Residuals}   
    \begin{itemize}
        \item The expression has derivatives of the value function 
        \item We want to have a measure that is easier to interpret. 
        \item Easy to show \alb{(envelope condition)} that \begin{align*}
            V^\prime\of{k,z} = \bp{z f^\prime\of{k} + 1 -\d} u^\prime\of{c} 
        \end{align*}
        where $c$ is the \al{optimal} consumption.
        \item Calculate \begin{align*}
            \Hc\of{k,z} \coloneq c - \bp{u^\prime}^{-1}\of{ \b \sum_{z^\prime \in \text{Z}} P\of{z,z^\prime} z^\prime f^\prime\of{k^\prime} u^\prime\of{c^\prime}}.
        \end{align*}
      
        \end{itemize}
    \end{frame}   

\begin{frame}{Residuals}   
    \begin{itemize}
        \item \al{Euler equation residuals} are \begin{align*}
            \Rc\of{k,z} \coloneq \frac{\Hc\of{k,z}}{c\of{k,z}}.
        \end{align*}
        \item Interpretation: agents make $ \Rc\of{k,z}\%$ mistakes when choosing $c$.
        \item Example: if $\Rc\of{k,z}$ is 0.01, then agents spend 1 dollar per 100 dollars "incorrectly".
        \item Bounded rationality interpretation. 

        \end{itemize}
    \end{frame}   

    \begin{frame}{Residuals}   
        \begin{itemize}
            \item Euler equation residuals should be quite small.
            \item They should be even smaller for points $\bp{k,z}$ that are visited frequently.
            \item You can immediately see the problem with a grid that is too coarse: grid points are unlikely to correspond to the exact maximizer $k^\prime$. 
            \end{itemize}
        \end{frame}   


        \begin{frame}
            \heading{Maximization}
            \end{frame}

    \begin{frame}{Bellman equation}   
        \begin{itemize}
            \item Recall the Bellman equation  
            \begin{align*}
                V\of{k,z} = \max_{k^\prime \in \G\of{k,z}} \bc{u\of{z f\of{k} + \bp{1-\d}k - k^\prime} + \b \sum_{z^\prime \in \text{Z}} P\of{z,z^\prime} V\of{k^\prime,z^\prime} },
            \end{align*}
            \item Solve the maximization problem directly using techniques we learned when we talked about optimization.
            \item We do not need to assume that $k^\prime$ belongs to the grid. 
            \item This gives us hope to finding the exact maximizer $k^\prime$.
            \item \al{Problem:} we do not know the exact value of $V$ at points that do not belong to the grid.
            
        \end{itemize}
        \end{frame}  

    \begin{frame}{Approximation}   
        \begin{itemize}
            \item We will use an \al{approximation} of $V$ at points that do not belong to the grid, $\hat{V}$. 
            \item Continuous piecewise linear interpolation is a safe choice.
            \item Then for each $k$ and $z$ we have to solve for $k^\prime$ using \begin{align*}
                \max_{k^\prime \in \G\of{k,z}} u\of{z f\of{k} + \bp{1-\d}k - k^\prime} + \b \sum_{z^\prime \in \text{Z}} P\of{z,z^\prime} \hat{V}\of{k^\prime,z^\prime} .
            \end{align*}
        \end{itemize}
        \end{frame}  

        \begin{frame}{Modification}   
            \begin{itemize}
                \item In the above we have \begin{align*}
                   \sum_{z^\prime \in \text{Z}} P\of{z,z^\prime} \hat{V}\of{k^\prime,z^\prime} 
                \end{align*}
                \item We can approximate $V$ and then calculate the sum (the expected value)
                \item It is usually cheaper to approximate the expected value directly.
                \item Let \begin{align*} W\of{k^\prime,z}\coloneq \sum_{z^\prime \in \text{Z}} P\of{z,z^\prime} V\prime\of{k^\prime,z^\prime} \end{align*}
                \item Obtain $\hat{W}\of{k^\prime,z}$ and then solve \begin{align*}
                    \max_{k^\prime \in \G\of{k,z}} u\of{z f\of{k} + \bp{1-\d}k - k^\prime} + \b \hat{W}\of{k^\prime,z}.
                \end{align*}
            \end{itemize}
            \end{frame}  
    
    \begin{frame}{Summary}   
        \begin{itemize}
            \item Start with some grid, as in the previous section.
            \item Instead of finding $k^\prime$ that belongs to the grid (by comparing continuation values directly), solve the maximization problem directly.
            \item This requires using some optimizer for each $k$ and $z$.
            \item This requires knowing $\hat{V}$ at points that do not belong to the grid.
            \item Important: $\hat{V}$ must preserve the concavity of $V$.
        \end{itemize}
        \end{frame}     

    \begin{frame}
        \heading{First order condition}
        \end{frame}

\begin{frame}{First order condition}   
    \begin{itemize}
        \item The maximization problem 
        \begin{align*}
            \max_{k^\prime \in \G\of{k,z}} u\of{z f\of{k} + \bp{1-\d}k - k^\prime} + \b \sum_{z^\prime \in \text{Z}} P\of{z,z^\prime} V\of{k^\prime,z^\prime},
        \end{align*} has a first order condition             \begin{align*}
            - u^\prime\of{z f\of{k} + \bp{1-\d}k - k^\prime} + \b \sum_{z^\prime \in \text{Z}} P\of{z,z^\prime} V^\prime\of{k^\prime,z^\prime} = 0.
        \end{align*}
        \item Instead of solving the maximization problem directly, we can solve the first order condition (a nonlinear equation) for $k^\prime$.
        \item To find it, we can use for example the \al{bisection method} or \al{Newton's method}.
    \end{itemize}
    \end{frame}  


    \begin{frame}{First order condition}   
        \begin{itemize}
            \item Same comments as with the maximization problem apply,
            \item We need to approximate $V^\prime$ at points that do not belong to the grid.
            \item We can do it by approximating $V$ and then calculating the derivative or by approximating the derivative directly.
            \item We can also approximate $W$ or $W^\prime$ instead.
        \end{itemize}
        \end{frame}  

    \begin{frame}
        \heading{Endogenous grid method}
        \end{frame}

        \begin{frame}{First order condition}   
            \begin{itemize}
                \item Recall the idea of the previous approach: fix a grid for $k$ and $z$ and then solve the nonlinear equation for $k^\prime$.
                \item The nonlinear equation is the most costly part. 
                \item In some cases we can avoid it by using the \al{endogenous grid method}.
            \end{itemize}
            \end{frame}             

            \begin{frame}{Endogenous grid method}   
                \begin{itemize}
                    \item The first order condition was 
                    \begin{align*}
                        u^\prime\of{z f\of{k} + \bp{1-\d}k - k^\prime} = \b W^\prime\of{k^\prime,z}
                    \end{align*}
                    and we wanted to find $k^\prime$ as a function of $k$ and $z$.
                    \item The endogenous grid method (by Chris Carroll) inverts the logic. 
                    \item Suppose you want to go to some $k^\prime$ given $z$. What $k$ do you need to start with?
                \end{itemize}
                \end{frame}       


            \begin{frame}{Endogenous grid method}   
                \begin{itemize}
                    \item Start with a grid for \al{$k^\prime$} and $z$.
                    \item Invert the first order condition as 
                    \begin{align*}
                        z f\of{k} + \bp{1-\d}k - k^\prime = (u^\prime)^{-1} \of{\b W^\prime\of{k^\prime,z}}
                    \end{align*}
                    and solve for $k$.
                    \item Although you might need to use a nonlinear solver, you do not need to evaluate $W^\prime$ repeatedly.
                    \item Now we know $k$ as a function of $k^\prime$ and $z$.
                    \item After we did it for all $k^\prime$ and $z$, we have an endogenous grid for $k$.
                \end{itemize}
                \end{frame}       

            \begin{frame}{Endogenous grid method}   
                \begin{itemize}
                    \item We now know $k\of{k^\prime,z}$ for each $k^\prime$ and $z$ in an \alb{exogenous} grid. 
                    \item We now use interpolation to get $k^\prime\of{k,z}$ for each $k$ and $z$ in the same \alb{exogenous} grid. 
                    \item After that we have $V$ for each (k,z) in the \alb{exogenous} grid.
                    \item Get $W$ and repeat the process.
                \end{itemize}
                \end{frame}       

\end{document}